%%% Clinic Statement of Work Template
%%%
%%% C.M. Connelly <cmc@math.hmc.edu>
%%%
%%%  $Id: statement-of-work-template.tex 353 2010-08-23 23:47:44Z cmc $


%%% !!! HMC STUDENTS SHOULD REMOVE THE FOLLOWING COPYRIGHT NOTICE FROM
%%% !!! FINAL SUBMISSIONS.

%%% Copyright (C) 2004-2010 Department of Mathematics, Harvey Mudd College.
%%%
%%% This file is part of the hmcclinic class document provided to
%%% HMC mathematics students.
%%%
%%% See the COPYING document, which should accompany this
%%% distribution, for information about distribution and
%%% modification of the document and its components.

%%% !!! END COPYRIGHT NOTICE.


%%% Clinic reports use the clinic class, which should be located
%%% somewhere in TeX's search path.

%%% For your ``statement of work'' (or ``work statement''), specify
%%% the ``proposal'' document-class option to the hmcclinic class.
\documentclass[proposal]{hmcclinic}

%%% The major difference between the statement of work and a midyear
%%% or final report is that the statement of work is typeset as an
%%% article, which means that the highest level of structural
%%% division available to you is section rather than chapter.

%%% There are also some changes in pagination styles and content
%%% that reflect the briefer nature of the proposal.  For example,
%%% in the longer reports, you use \frontmatter, \mainmatter, and
%%% \backmatter to separate some sections of the report from
%%% others.  In the statement of work, you don't need those
%%% commands, as no such division is necessary.

%%% Other packages needed by your document may be loaded here.
% \usepackage{url}              % For formatting URLs and other web or
                                % file references.

%%% Provide additional context around errors. 
\setcounter{errorcontextlines}{1000}


%%% Information about this document.

%%% I find it most useful to put identifying information about a
%%% document near the top of the preamble.  Technically, this
%%% information must precede the \maketitle command, which often
%%% appears immediately after the beginning of the document 
%%% environment.  Placing it near the top of the document makes it
%%% easier to identify the document, and keeps it out from getting
%%% mixed up with the real meat of the document.

%%% We use the same set of commands for specifying information about
%%% the people involved with the project that are used in the longer
%%% reports, so you can copy most of this information directly into
%%% your midyear and final reports.

%%% So, some questions.

%% What is the name of the company or organization sponsoring your project?
\sponsor{Harvey Mudd College}

%% What is the title of your report?
\title{Statement of Work}

%% Who are the authors of the report (your team members)?  (Separate
%% names with \and.)
\author{Brett Collins (Project Manager) \and Alex Gruver \and Ellen Hui \and
Tyler Marklyn}

%% What is your faculty advisor's name?  (Again, separate names with
%% \and, if necessary.)
\advisor{Jeff Amelang}

%% Liaison's name or names?
\liaison{Carter Edwards \and Robert Hoekstra}

%% Did you have an outside consultant help you with this project?  Put
%% their names in the \consultant command.

%%% End of information section.

%%% New commands and environments.

%%% You can define your own commands and environments here.  If you
%%% have a lot of material here, you might want to consider splitting
%%% the commands and environments into a separate ``style'' file that
%%% you load with \usepackage.

\newcommand{\coolcommand}[1]{#1 is cool.} % Lets everyone know that
                                % the person or thing that you provide
                                % as the argument to the command is
                                % cool.


%%% Some theorem-like command definitions.

%%% The \newtheorem command comes from the amsthm package.  That
%%% package is loaded by the class file.

%%% Note that these definitions have changed from the version in the
%%% sample report document by dropping the ``within'' argument.  See
%%% Gratzer's _Math into LaTeX_ or the AMS-LaTeX documentation for
%%% more details.

% \newtheorem{thm}{Theorem}
% \newtheorem{Theo1}{Theorem}
% \newtheorem{Theo2}{Theorem}
% \newtheorem{Lemma}{Lemma}


%%% If you find that some words in your document are being hyphenated
%%% incorrectly, you can specify the correct hyphenation using the
%%% \hyphenation command.  Note that words are separated by
%%% whitespace, as shown below.

\hyphenation{ap-pen-dix wer-ther-i-an}


%%% The start of the document!

%% The document environment is the main environment in any LaTeX
%% document.  It contains other environments, as well as your text.

\begin{document}

%%% In a longer document (such as your midterm and final reports),
%%% you would have separate \frontmatter, \mainmatter, and
%%% \backmatter commands to define some large chunks of your
%%% document.  For the Statement of Work, which is a short document,
%%% we don't need these commands.

%%% Your Statement of Work begins with a title page.  The title page
%%% is formatted by commands in the document class file, so you
%%% don't need to worry about what it looks like -- just putting the
%%% \maketitle command in your document (and filling in the necessary
%%% information for the identification commands above) is enough.
\maketitle

%%% In a longer document or an article being submitted to a journal
%%% or conference, you would probably have an abstract that
%%% summarized the purpose of the document.  We don't need that for
%%% a Statement of Work.

%%% Similarly, in longer documents you would probably have commands
%%% to include a table of contents and lists of figures or tables.
%%% For a short document such as the Statement of Work, we don't
%%% need these commands.


%%% Content.

%%% For smaller documents---especially those you're writing by
%%% yourself---you might write your entire report using a single LaTeX
%%% source file.  For larger documents, we recommend that you split
%%% the source file into several separate, smaller files.  The smaller
%%% files are ``included'' into your main, or ``master'' document
%%% using \include commands.  See the template file for the Clinic
%%% reports for more details on how to split a LaTeX project into
%%% smaller files.


%%% The body of your Statement of Work should appear here.  See
%%% Chapter 4 in _The Mathematics Clinic in Brief: A Handbook_ for
%%% more details on what you should include in a Statement of Work.

\section{Problem Statement}

Sandia National Laboratories currently has many libraries and legacy code that do not utilize new multicore architectures. Using their Kokkos library, which was created to make it possible and easy to port code to new architectures (specifically multicore architectures), we will make their tensor contractions library and, as time permits, other libraries thread scalable.

% TODO

\section{Background}

Sandia National Laboratories is a federally funded research and development
center owned and managed by the Sandia Corporation.  The laboratory's primary
focus is the maintenance, management, and development of the United States'
nuclear arsenal.  Sandia also performs research in the fields of supercomputing
and scientific computing.  The Trilinos Project, developed and maintained by
Sandia, is a collection of open source libraries intended for use in large-scale
scientific applications.

One of the packages in Trilinos is Kokkos, which is designed to aid portability
and performance in software written for manycore architectures.  Well-designed
and well-implemented parallelism can yield significant speedup in certain
processing-heavy applications, including many in the scientific computing
domain.  However, optimizing an application to take advantage of many processors
is highly dependent on the architecture on which the application is to run.  In
addition, highly parallel code may suffer from race conditions between threads,
causing either incorrect results or slower runtimes resulting from the steps
necessary to ensure correctness.

Kokkos attempts to mitigate these issues by providing allowing programmers to
write their code once, and compile for optimization on a variety of
manycore architectures.  Kokkos also allows users to write thread-scalable
software by providing an API for using fast, non-locking builtins.

However, the Kokkos package is still new and relatively untested.  No
large-scale projects have yet been written using Kokkos.  Some small kernels
have been rewritten using Kokkos, but there are still many kernels in Trilinos
that would benefit from increased thread-scalable parallelization.

\section{Goal}

The goal of this clinic is to rewrite several Trilinos kernels to be efficient
and thread-scalable on manycore architectures, using Kokkos.  These redesigned
and reimplemented kernels will be integrated into Trilinos' production code, as
both a performance improvement and as a further test for Kokkos.

The two kernels we plan to work on are Intrepid, a tensor manipulation library,
and
% (TODO fill this in later.)

\section{Objectives}

As stated, the primary goal of this clinic is to transform Sandia's tensor manipulation and $X$ library to be thread scalable using Kokkos. This goal can be broken down into a couple different objectives. First, our team needs to get familiar with Sandia's Kokkos library. Learning the syntax, benefits, and disadvantages of Kokkos will allow us to optimize the libraries as much as possible and why it is our first objective. Our second major objective will apply the knowledge that is learned from completing the first objective: rewritting and refactoring Sandia's tensor contraction to be thread scalable using Kokkos. The third objective, which is essentially the same as the second, is to make the $x$ library thread scalable also. All of these objectives are big and very time consuming, which is why they will be broken down more in the Tasks section.

% TODO

\subsection{Optional Objectives}

The optional objectives presented in this section are to be done if time permits. Firstly, if the team has finished, or has made significant progess towards making Sandia's tensor manipulation library thread scalable, then a seminar on how to use Kokkos will be given to a group of Sandia's employees on the site visit. After making the tensor manipulation and $x$ libraries thead-scalable, there are a couple different objectives that can be pursued, depending on the priority given by the liaisons.The first possible objective is to continue making more of Sandia's libraries thread-scalable. This benefits Sandia by making more code run faster. While another possibility is to code versions of the tensor manipulation and $x$ libraries in CUDA, or another framework such as OpenCL, in order to compare their performance with Kokkos' performance. The performance tests can show the strengths and weaknesses of Kokkos and give information that is useful to the developers of Kokkos. The performance tests and examples of Kokkos' simplicity can be useful examples for helping Sandia try to integrate Kokkos into the C-standard.

% TODO

\section{Tasks}

Our first task for this semester is to acquire a computer with suitible
computational capabilities. In practice, this translates to either an NVIDIA
K20X or an NVIDIA Titan tesla card. This will allow us to test the thread
scalability of our programs and insure that they meet Sandia's needs. 

After this, we will try to familiarize ourselves with massively multithreaded
programming by working through a series of example problems in a veriety of
environments. So far, we have identified numerical integration, matrix
multiplication, and histogram calculations as likely candidates for practice
problems. We will try to implement these problems in Intel's Threaded Building
Blocks, CUDA, and OpenMP. 

After familiarizing ourselves with TBB, CUDA and OpenMP, we will have a
background to understand the ideal way to write code using Kokkos. At this
point, we will begin to write solutions to our practice problems using Kokkos.
We can then compare the speedup gains on Kokkos compared to the other
libraries, and also the ease of writing thread scalable code using Kokkos
comapred ot the other libraries. This will be good practice because ideally, we
will also test a few of the kernels we write so that we can see differences is
ease of coding and speedup in them as well.

At this point, we will shift our focus to our first kernel.
While we have a significant Mathematics background none of us have any
experience with tensor contraction.  We will need to explore the main
algorithms used for tensor contraction and determine if there are existing
procedures for parallelizing the process. The original Kokkos source code,
along with other materials should help us with this. 

Once we are familiar with Kokkos, and we feel we have enough of a background
with the technologies, we will begin working on improving the tensor
contraction library. It's hard to say how difficult this will be at this point,
but we hope that we will be able to present our progress at the end of semester
site visit at Sandia. 

After the tensor contraction library ...... TODO
%%% Appendices.

%%% For your Statement of Work, you probably won't have any
%%% appendices, but you could include some if you really needed to.

%%% The appendices are delineated with the \appendix command.
%%% Individual appendices are begun with the standard \chapter or
%%% \section commands.  In our example, we'll \include them just as we
%%% did other chapters.

%%% Even in a relatively short document such as your statement of
%%% work, you might need to have appendices.  If so, uncomment
%%% the \appendix command and add them below (remember, the
%%% top-level structural command in this format is section).

% \appendix


%%% Bibliography.

%%% BibTeX is the tool to use for citations and layout of your
%%% bibliography.  Instead of having to type ``[5]'' or ``(Jones,
%%% 1968)'' (and keep track of which citation is which and renumber
%%% them as you add more references to your bibilography), you use
%%% special commands that allow BibTeX and LaTeX to automatically put
%%% the correct information in the right place.

%%% Section 5.6 in _The Mathematics Clinic in Brief: A Handbook_,
%%% talks about using BibTeX to format your bibliography and
%%% citations.

%%% Depending on your field, it may or may not be appropriate to list
%%% references for which you haven't included specific citations.  If
%%% your field sanctions such practices, or if you just want to get an
%%% idea of what you have in your bibliography file, you can include
%%% everything with the \nocite{*} command.
\nocite{*} 


%%% The appearance of your bibliography and citations in your text are
%%% defined by a combination of any bibliography-related LaTeX
%%% packages (such as natbib, harvard, or chicago) and the particular
%%% bibliography style file that you load with the \bibliographystyle
%%% command.  Bibliography-style files end in .bst; you can find them
%%% by searching your file system using whatever tools you have for
%%% doing searches.  (On most modern Unices, ``locate .bst'' will give
%%% you an idea of what's available.)

\bibliographystyle{hmcmath}

%%% The particular bibliography data file or files that you want to
%%% use are specified with the \bibliography file.  Multiple files are
%%% separated by commas.

%%% You might want to use multiple bibliography (or ``bib'') files if
%%% you had a master bib file containing references you use again and
%%% again, and another containing only records for references for a
%%% particular project.

%%% Many people create a single, large bib file that they use for
%%% everything they write.  That approach requires you to \cite every
%%% reference that you want to use in your document -- using
%%% \nocite{*} with a huge bibliography database will give you a large
%%% bibliography containing many references you haven't consulted for
%%% your particular document!

\bibliography{sample}


%%% Glossary or Index.

%%% Having a glossary or index in a statement of work is overkill.
%%% Just define your terms in the text and you'll be fine.

\end{document}

